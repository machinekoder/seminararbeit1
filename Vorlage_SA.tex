% !TEX encoding = IsoLatin2  % notwendige Zeile f"ur Mac-Benutzer (muss als Kommentar stehen); Windows-Benutzer k"onnen
%die Zeile l"oschen.

% LaTeX-Vorlage Version 3.1,  Juli 2011
% erstellt von Dr. Andreas Drauschke (andreas.drauschke@technikum-wien.at) und Dr. Susanne Teschl (susanne.teschl@technikum-wien.at)
% geringf"ugig adaptiert von Harald Stockinger (harald.stockinger@technikum-wien.at)


\documentclass[11pt,a4paper,bibtotoc,oneside]{scrbook}
% F"ur kurze Arbeiten w"are auch die Dokumentklasse "scrartcl" ausreichend. In diesem Fall ist "section" die h"ochste Ebene ("chapter" gibt es dann nicht).
% \documentclass[a4paper,bibtotoc,oneside]{scrartcl}


%Zum Verlinken des inhaltsverzeichniss & co
\usepackage[colorlinks=false]{hyperref}
% %hyperref setup`%
\hypersetup{pdfborder=0 0 0}
%       colorlinks=false,
%       citecolor=Violet,
% %         linkcolor=Green}
\usepackage[utf8x]{inputenc}
% deutsche Anpassungen
% \usepackage[ansinew]{inputenc}
\usepackage[T1]{fontenc}
\usepackage[ngerman]{babel}
% mathematische Symbole
\usepackage{amsmath,amssymb,amsfonts,amstext}
\usepackage{xcolor}
\usepackage{calc}
\usepackage{caption}
%tabellen

% Kopfzeilen frei gestaltbar
\usepackage{fancyhdr}
\lfoot[\fancyplain{asdf}{}]{\fancyplain{}{}}
\rfoot[\fancyplain{}{}]{\fancyplain{}{}}
\cfoot[\fancyplain{}{\footnotesize\thepage}]{\fancyplain{}{\footnotesize\thepage}}
\lhead[\fancyplain{}{\footnotesize\nouppercase\leftmark}]{\fancyplain{}{}}
\chead{}
\rhead[\fancyplain{}{}]{\fancyplain{}{\footnotesize\nouppercase\sc\leftmark}}

% Farben im Dokument m"oglich
\usepackage{color}

% Schriftart Helvetica
\usepackage{helvet}
\renewcommand{\familydefault}{cmss}

% Graphiken einbinden: hier f"ur pdflatex
\usepackage[pdftex]{graphicx}
% Um pdf einzufügen
\usepackage{pdfpages}
\usepackage{array}

% H"ohe und Breite des Textk"orpers etwas gr"osser definieren
\setlength{\textheight}{260mm}
\setlength{\textwidth}{1.05\textwidth}

% weniger Warnungen wegen "uberf"ullter Boxen
\tolerance = 9999
\sloppy

% Anpassung einiger "Uberschriften
\renewcommand\figurename{Abbildung}
\renewcommand\tablename{Tabelle}

% %footers and headers
% % \usepackage{fancyhdr}
% % \pagestyle{fancy}
% \lhead{\studiumshort}
% \chead{}
% \rhead{\fachnameshort}
% \lfoot{\teilnehmeronenachname, \teilnehmertwonachname}
% % \cfoot{erstellt am: \today}
% \rfoot{\thepage}
% \renewcommand{\headrulewidth}{0.5pt}
% \renewcommand{\footrulewidth}{0.5pt}
\usepackage{geometry}
\begin{document}
% Kopf- und Fusszeilen initiieren
\pagestyle{fancy}

% Deckblatt:
\thispagestyle{empty}
\begin{picture}(0,0)
\color{white}\sffamily
\put(-101,-390){\includegraphics[width=1.002\paperwidth]{./picture/LPS_2011.pdf}}
\put(220,-670){\includegraphics[width=0.5\textwidth]{./picture/FHTW_Logo_4c.pdf}}
\put(-30, -20){\bfseries\huge SEMINARARBEIT}
% Titel des Studienganges einf"ugen:
\put(-30,-50){\Large im Studiengang BEL3}
% Titel der Lehrveranstaltung einf"ugen:
\put(-30,-70){\Large Lehrveranstaltung Audiotechnik}
\color{black}
% Titel der Arbeit einf"ugen:
% Die Minipage wird gesetzt, damit auch mehrzeilige Titel m"oglich werden.
\put(-32,-350){
\begin{minipage}{13cm}
\bfseries\huge D-Verstärker
\end{minipage}
}
% Name der Autorin/des Autors eingeben:
% Personenkennzeichen der Autorin/des Autors eingeben:
\put(-30,-450){\large Ausgeführt von:\ Christian Schwarzgruber}
\put(+54,-470){\large \ Alexander Rössler}
\put(-30,-490){\large Matrikelnummer: 1110254027}
\put(+63,-510){\large 1110254020}
% Name der Begutachterin/des Begutachters eingeben:
\put(-30,-550){\large Begutachter: Michael Windisch}
\put(-30,-590){\large Wien, \today} % das Datum des letzten Kompilierens wird automatisch eingesetzt
\end{picture}
\savegeometry{GEO1}
\newpage

\tableofcontents\thispagestyle{empty}
\newpage

\setcounter{page}{1}
\newgeometry{top=10mm, left=20mm, right=20mm, bottom=25mm, %Package must be first otherwise headerline are
headsep=8mm, footskip=8mm}
\savegeometry{GEO2}
% Falls die Kapitel"uberschriften zu lang f"ur die Kopfzeile oder das Inhaltsverzeichnis sind, so erzielt man
% dort Kurzformen der Kapitelbezeichnungen mittels:
% \chapter[Kurzform]{Lange "Uberschrift}
\chapter[Recherche über Klasse D-Verstärker]{Recherche über Klasse D-Verstärker}

%=====================================================================================================================%
\section{Recherche}
Es gibt unzählige Möglichkeiten einen Klasse D-Verstärker zu
implementieren, doch besitzen sie allesamt die selben
Grundbauelemente, wie Schmiddtrigger und Dreiecksgenerator.\\ Oft wird eine analoge PW M-Steuerung verwendet um, dass
Signal mit nur zwei Spannungszuständen zu erzeugen. Es gibt noch verschiedene andere analoge und digitale Verfahren
bzw. Verfeinerungen, wie Pulsfrequenzmodulation, Simga-Modulation oder
Sliding-Mode-Regelung.\textcolor{blue}{\cite{digWik}}
Bei der Recherche sind wir auf eine gute Anleitung gestoßen, die auch noch dazu sehr gut erklärt
ist.\textcolor{blue}{\cite{cae}}
Im Internet findet man unzählige Anleitungen über den Aufbau von Klasse-D-Verstärker, manche sind sehr umfangreich und
detailliert erklärt.
\section{Funktionsweise}
Das eingangs Signal wird zunächst in ein Digitales Signal umgewandelt, meist geschieht das mit einer PWM-Steuerung, wie
oben beschrieben. Dieses Signal ist ist nun viel Hochfrequenter und kann so Verlustarm verstärkt werden. Die PWM kennt
nur zwei Zustände 1 und 0, sprich High und Low.
\section{Verwendungsmöglichkeiten von Klasse D-Verstärker}

\section{Alternativen}
\subsection{A-Verstärker}
Klasse a besteht aus einer Transistor Schaltung mit einem Transistor. Der Arbeitspunkt liegt
in der Mitte. Diese Klasse wird heute kaum noch verwendet Grund hierfür ist der geringe
Wirkungsgrad von nur 6.25\%, unabhängig von der Last. Das heißt wenn ein Verstärker der Klasse-A ein Ausgangsleistung
von 10 Watt liefert wandelt es ständig 60 Watt in Wärmeenergie um. \textcolor{blue}{\cite{clA}}
\subsection{B-Verstärker}

\textcolor{blue}{\cite{clA}}
\subsection{AB-Verstärker}
\textcolor{blue}{\cite{clA}}
\subsection{Weitere Verstärker Typen}
\textcolor{blue}{\cite{clA}}
%=====================================================================================================================%
\chapter{Dreiksgenerator \& Komperatorschaltung}
\section{Dreiksgenerator}
\subsection{Schaltungsaufbau}

\subsection{Dimensionierung}
\subsection{Funktionsweise}
\section{Komperatorschaltung}
\subsection{Schaltungsaufbau}
\subsection{Dimensionierung}
\subsection{Funktionsweise}

\chapter{OPV Vergleich}
\section{OPV-LM324}
Der OPV-LM324 hat eine sehr geringe Slew Rate von nur 0.5 $V/µs$ dies ist der Grund warum das Ausgangssignal nicht dem
Eingangssignal folgen kann. Siehe Abbildung \textcolor{blue}{\ref{lm324}}
\subsection{Simulation}
    \begin{figure}[h]
    \centering
        \includegraphics[width=300pt]{./picture/LM324_triangle.png}
        % LM324_triangle.png: 0x0 pixel, 250dpi, 0.00x0.00 cm, bb=
        \caption{\label{lm324}{Ausgangssignal LM324}}
    \end{figure}
    % LM324_triangle.png: 0x0 pixel, 250dpi, 0.00x0.00 cm, bb=
\section{OPV-OPA350}
Der OPV-OPA350 hat eine sehr hohe Slew Rate von 22 $V/µs$ welche sich bemerkbar macht. Das Ausgangssignal kann dem
Eingangssignal folgen. Siehe Abbildung \textcolor{blue}{\ref{opa324}}
\subsection{Simulation}
\begin{figure}[h]
\centering
    \includegraphics[width=300pt]{./picture/OPA350_triangle.png}
    % OPA350_triangle.png: 0x0 pixel, 250dpi, 0.00x0.00 cm, bb=
    \caption{\label{opa324}Ausgangssignal OPA350}
\end{figure}
\chapter{Klasse-D-Verstärker Schaltung}


\chapter{Seminarbuch}
\begin{table}[htbp]
  \centering
  \captionsetup{margin=1pt,font=small,labelfont=bf}
  \caption{Tätigkeit}
    \begin{tabular}{| c | c| }\hline
    {\bf Arbeitstag} &{\bf Tätigkeit} \\\hline
    \hline
    1. Tag   & Recherche im Internet über D-Verstärker  \\
    2. Tag   & Dimensionierung der Dreiecksgenerators und Simulation \\
    3. Tag   & Durchführung Übung 3 Signalkennwerte Messung  \\
    4. Tag   & Durchführung Übung 3 Signalkennwerte Messung  \\
    4. Tag   & Durchführung Übung 3 Signalkennwerte Messung  \\
    \hline
    \end{tabular}%
  \label{tab:addlabel}%
\end{table}%
\begin{table}[htbp]
  \centering
    \captionsetup{margin=1pt,font=small,labelfont=bf}
      \caption{Arbeitsaufwand}
      \begin{tabular}{| c | c |}\hline
      {\bf Arbeitszeit(h)} &{\bf Dokumentation(h)} \\\hline
      \hline
        2   & 2 \\
        2   & 2 \\
        4   & 7 \\
      \hline
        \textbf{12}   & \textbf{14} \\
      \hline
      \end{tabular}%
    \label{tab:addlabel}%
\end{table}%





\noindent
Hier ist ein Hyperlink auf die  \href{http://www.technikum-wien.at}{Homepage} der FH Technikum Wien. Email-Adressen k"onnen so verlinkt werden: \href{mailto:homer.simpson@springfield.com}{\texttt{homer.simpson@springfield.com}}\\

\noindent
In der Bibliothek der Fachhochschule Technikum Wien gibt es verschiedene einf"uhrende B"ucher zum Thema \glqq \LaTeX \grqq, zum Beispiel \cite{kop05}, \cite{wil06} oder \cite{mgb+05d} (deutsche Version) bzw. \cite{mgb+04e} (englische Version). Empfehlenswerte Skripten f"ur \LaTeX-Einsteiger sind z.B. \cite{mj00} und \cite{mj95}. Sie sind frei im Internet verf"ugbar.

\restoregeometry{GEO1}
% Literaturverzeichnis
% Das Literaturverzeichnis kann auch nach einem allf"alligen Anhang positiioniert werden (siehe "`Leitfaden f"ur Bachelor- und Diplomarbeiten"', Version 2.0, Abschnitt 2.9).

% M"oglichkeit 1: Erzeugung des Literaturverzeichnisses mit BibTeX:
% Die Quellen sind in der Datei *.bib (hier Literatur.bib) einzugeben. Danach muss diese Vorlage einmal geTeXt werden,
dann BibTeX angewendet werden und
% anschliessend nochmals zweimal geTeXt werden.
% Im Text erfolgt die Zitierung mit dem Anker-Schl"usselwort, z.B. \cite{kop05}.
\bibliographystyle{IEEEtran}
\bibliography{Literatur}

% M"oglichkeit 2: Erzeugung eines Literaturverzeichnisses ohne BibTeX:
%\begin{thebibliography}{99}
%\bibitem[kop05]{kop05}
%H.~Kopka, {\em LaTeX, Band 1: Einf"uhrung}, Pearson Studium, M"unchen, 3.~Auflage, 2005.
%\bibitem[knu98]{knu98}
%F.~Mittelbach, M.~Goossens, J.~Braams, D.~Carlisle, and Ch. Rowley, {\em The LaTeX Companion},
%Addison-Wesley, 2nd edition, 2004.
%\end{thebibliography}

% Abbildungsverzeichnis
\listoffigures
\addcontentsline{toc}{chapter}{Abbildungsverzeichnis} % f"ugt den Eintrag "Abbildungsverzeichnis" im Inhaltsverzeichnis hinzu
\newpage

% Tabellenverzeichnis
\listoftables
\addcontentsline{toc}{chapter}{Tabellenverzeichnis} % f"ugt den Eintrag "Tabellenverzeichnis" im Inhaltsverzeichnis hinzu
\newpage

% Abk"urzungsverzeichnis
% Bei Verwendung der Dokumentklasse "scrartcl" ist der Befehlt \addchap{Abk"urzungsverzeichnis} durch
% \addsec{Abk"urzungsverzeichnis} zu ersetzen
\addchap{Abk"urzungsverzeichnis}
\hspace{-17mm}\begin{tabular}{>{\raggedleft}p{0.2\linewidth} p{0.75\linewidth} p{0.1\linewidth}}
www & World Wide Web \\
URL & Uniform Resource Locator
\end{tabular}

% Anh"ange
\begin{appendix}
\chapter[Erster Anhang]{Entwicklung und Aufbau eines Klasse D-Verstärkers}
% \lhead{}
% \lhead{\textcolor{blue}{\url{www.widatec.com/}}}\label{Anhang1}
% \includepdf[pages={1-24},addtotoc={{24},{section},{2},{Entwicklung und Aufbau eines Klasse
% D-Verstärkers},{D-Verstärker}}, pagecommand={\thispagestyle{fancy}},noautoscale=true,width=1.1\textwidth,offset=0cm
% 1cm]{/home/christian/FH/3.Semester/ATK/Projekt/seminararbeit1/unterlagen/CAE.pdf}
% \lhead{\studiumshort}
%

\chapter[Zweiter Anhang]{"Uberschrift des zweiten Anhangs}

Text Text Text Text Text Text Text Text Text Text Text Text Text Text Text Text Text Text Text Text Text Text Text Text ...

\end{appendix}

\end{document}
